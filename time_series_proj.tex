\documentclass[12pt,a4paper]{book}
%\documentclass[12pt,openright,twoside,a4paper,brazil]{abntex2}
\usepackage[utf8]{inputenc}
\usepackage[T1]{fontenc}
\usepackage{amsmath}
\usepackage{amssymb}
\usepackage{graphicx}
\usepackage[brazilian]{babel}
\usepackage{glossaries}
\makeglossaries
\newglossaryentry{latex}
{
	name=latex,
	description={LaTeX (short for Lamport TeX) is a document preparation system. The user has to think about only the content to put in the document and the software will take care of the formatting. }
}

\newglossaryentry{glsy}
{
	name=glossary,
	description={Acronyms and terms which are generally unknown or new to common readers.}
}

\newglossaryentry{ubuntu}
{
	name=ubuntu,
	description={it is a Linux Version}
}


\newglossaryentry{linux}
{
	name=linux,
	description={it is a operational system}
}

\newacronym{ar}{AR}{Auto Regressive Time Serie}

\newacronym{ma}{MA}{Moving Average}

\newacronym{hw}{HW}{Holt-Winters}

%\newglossaryentry{latex}
%{
%	name=latex,
%	description={LaTeX (short for Lamport TeX) is a document preparation system. The user has to think about only the content to put in the document and the software will take care of the formatting. }
%}
%
%\newglossaryentry{glsy}
%{
%	name=glossary,
%	description={Acronyms and terms which are generally unknown or new to common readers.}
%}


\title{TESTE DO \LaTeX: RETORNANDO ÀS RAÍZES}
\author{MARCOS FERREIRA}
\date{2023-01-24}



\begin{document}
	\maketitle
	\tableofcontents{} 
	\chapter{INTRODUÇÃO}
	\section{Alguns exemplos de Séries Temporais}
		Este é um teste rápido de \LaTeX 
	\section{Estrutura do Trabalho}
		Este trabalho divide-se nesta introdução, o capítulo 1 ,\cite{dinardo1997econometric,morettin2018analise}\cite{tripathi2000econometric,hamilton2020time},\cite{de2018tendencias,shumway2000time}, \cite{fischer1982series}
	
	\chapter{PROCESSO ESTOCÁSTICO}
	\Gls{latex} is very useful and can use \gls{glsy}.
	
	\gls{ubuntu} is a \gls{linux} system.
	
	\acrlong{ar} is used to autoregressive time series equations.
	
	\acrshort{ma} is used for Moving average time series
	
	\acrshort{hw} is used for holt-winters
	
	\chapter{PROCESSOS ESTACIONÁRIOS}
	\printglossaries
	\bibliography{bibliografia}
	\bibliographystyle{ieeetr}	
	\clearpage
\end{document}



